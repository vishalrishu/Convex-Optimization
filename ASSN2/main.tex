\documentclass[fleqn]{article}
\usepackage[utf8]{inputenc}
\usepackage[pdftex]{graphicx}
\usepackage{grffile}
\usepackage{float}
\usepackage{amsmath}
\usepackage{amssymb}
\usepackage[margin=1.25in]{geometry}
\usepackage{subfig} 


\title{Assignment 1}
\author{Vishal Kumar, MIT2019090}
\date{}

\begin{document}

\maketitle

\section*{Question}
{\bf What is Lagrange dual problem? Explain weak and strong Duality? Give relevant equations.}
\\
{\bf Ans. -} 
\\
The Lagrange dual problem is obtained by forming the Lagrangian of a minimization problem by using non-negative Lagrange multipliers to add the constraints to the objective function, and then minimize the original objective function by solving for the primal variable values.
\\
\\
Let Optimization problem is:
\begin{align*}
minimise. \quad& f_0(x)\\
s.t. \quad& f_i(x) \le 0 \quad i=1,\ldots,m\\
\\
Lagrangian:\\
L(x,\lambda) &= f_0(x) + \sum_{i=1}^m \lambda_i f_i(x)
\end{align*}
$\lambda$ are Langrange multipliers.
\\
So, the supremum over langrange.
\begin{align*}
\max_{\lambda \geq 0} \quad& L(x,\lambda) = \max_{\lambda \geq 0} (f_0(x) + \sum_{i=1}^m \lambda_i f_i(x))
\end{align*}
\begin{align*}
\max_{\lambda \geq 0} \quad& L(x,\lambda) = \begin{cases}f_0(x) & f_i(x) \le 0\\ \infty & otherwise
\end{cases}
\end{align*}
So, the Primal form of optimisation problem is:
\begin{align*}
p^* = \min_{x}\max_{\lambda \geq 0}\quad& L(x,\lambda)
\end{align*}
Langrange dual function is:
\begin{align*}
g(\lambda) = \min_{x}\quad& L(x,\lambda)
\end{align*}
Now, the Langrangian dual problem is:
\begin{align*}
d^* = \max_{\lambda \geq 0} g(\lambda) = \max_{\lambda \geq 0}\min_{x}\quad& L(x,\lambda)
\end{align*}
Weak duality states that any feasible solution to the dual problem corresponds to an upper bound on any solution to the primal problem. i.e.,
\begin{align*}
p^* \geq d^*
\end{align*}
While, Strong duality occurs when the values of the optimal solutions to the primal problem and dual problem are always equal. i.e.,
\begin{align*}
p^* = d^*
\end{align*}
\end{document}

