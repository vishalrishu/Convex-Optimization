\documentclass[fleqn]{article}
\usepackage[utf8]{inputenc}
\usepackage[pdftex]{graphicx}
\usepackage{grffile}
\usepackage{float}
\usepackage{amsmath}
\usepackage{amssymb}
\usepackage[margin=1.25in]{geometry}
\usepackage{subfig} 
\usepackage{bm}
\usepackage{algorithm}
\usepackage[noend]{algpseudocode}

\title{Assignment 4}
\author{Vishal Kumar, MIT2019090}
\date{}

\begin{document}

\maketitle
\section*{Question}
{\bf Explain Newton’s method with equality constraints. Also explain Newton step with infeasible start.}
\\
{\bf Ans. -}
\\
Lets consider a problem with linear equality constraints.
\begin{align*}
\min_{x \in \mathcal{R}^N}\quad& f(x)\\
s.t. \quad& h(x) = 0
\end{align*}
Here, assume f(x) is convex and twice differentiable.
\\
Now, Let g(x) is gradient of f(x).
\\
Let H(x) is Hessian of f(x) and
\\
J(x) is Jacobian.
\\
for non-constraint probelm, Newton' Method is:
\begin{align*}
    x_+ = x + \Delta x_{nt}
\end{align*}
here, Newton's step is:
\begin{align*}
    \Delta x_{nt} = - H(x)^{-1} g(x)
\end{align*}
To solve $\Delta x_{nt}$, We have to solve this linear equation:
\[
\begin{bmatrix}
    H(x)       & J(x)^T \\
    J(x)       & 0 \\
\end{bmatrix}
\begin{bmatrix}
    \Delta x_{nt}        \\
    \lambda       \\
\end{bmatrix}
=
\begin{bmatrix}
    -g(x)\\
    -h(x)\\
\end{bmatrix}
\]
Assume that h(x) = Ax and solution is feasible for h(x) = 0, then linear system:
\[
\begin{bmatrix}
    H(x^k)       & A^T \\
    A       & 0 \\
\end{bmatrix}
\begin{bmatrix}
    \Delta x        \\
    \lambda       \\
\end{bmatrix}
=
\begin{bmatrix}
    -g\\
    0\\
\end{bmatrix}
\]
Let's suppose H(x) is strictly convex, and A is full rank matrix then linear system is solvable:
\[
\begin{bmatrix}
    \Delta x        \\
    \lambda       \\
\end{bmatrix}
=
\begin{bmatrix}
    H(x^k)       & A^T \\
    A       & 0 \\
\end{bmatrix}^{-1}
\begin{bmatrix}
    -g\\
    0\\
\end{bmatrix}
\]
This is Newton's step with feasible start equation.

\begin{algorithm}
\caption{Newton's method with equality constraints}
\begin{algorithmic}[1]
    \State given starting point x $\in$ $\bf{dom}$ f and Ax = b, tolerance $\epsilon$ > 0
    
    \State repeat 
    
    \State Compute the Newton step and decrement $\Delta x_{nt}, \lambda(x) $.
    
    \State Stopping criterion. quit if $\lambda^2/2 \le \epsilon$
    
    \State Line search. Choose step size t by backtracking line search.
    
    \State Update. x := x + t$\Delta x_{nt} $
\end{algorithmic}
\end{algorithm}

\newpage
{\bf Newton Step with Infeasible start:}
\\
\\
The Newton step of f(x) at an infeasible point x for the linear
equality constrained problem is given by the solution of:
\\
\[
\begin{bmatrix}
    H(x)       & A^T \\
    A       & 0 \\
\end{bmatrix}
\begin{bmatrix}
    \Delta x_{nt}       \\
    \lambda       \\
\end{bmatrix}
=
\begin{bmatrix}
    -g(x)\\
    Ax-b\\
\end{bmatrix}
\]
\\
Let x be the an infeasible point. Our goal is to find a step\\ $\Delta$ x s.t. x + $\Delta$x satisfies approximately the optimality condition. After linearization we get:
\begin{align*}
A(x + \Deltax) = b ,\\
g(x) + H(x)\Deltax + A^T \lambda = 0,
\end{align*}
i.e., the definition of the Newton step
\\
\\
These are the primal and dual conditions of Newton's method. We have to update x and $\lambda$ both in order to satisfy the optimal conditions i.e. we have to update till Ax = b.

\begin{algorithm}
\caption{Newton's method with equality constraints}
\begin{algorithmic}[1]
    \State given starting point x $\in$ $\bf{dom}$ f,$\lambda$ tolerance $\epsilon$ > 0, $\alpha \in (0, 1/2), \beta \in (0,1)$
    
    \State repeat 
    
    \State Compute primal and dual Newton steps $\Delta x_{nt}, \Delta \lambda_{nt} $
    \State Backtracking line search on $\parallel r \parallel_2$ 
    
    t:=1
    
    \textbf{while} $\parallel r(x + t\Delta x_{nt}, \lambda + t\Delta \lambda_{nt} \parallel_2 > (1 - \alpha t) \parallel r(x,\lambda) \parallel_2. \;\;\; t:=\beta t$
    
    \State Update, $x:= x+t\Delta x_{nt}, \lambda := \lambda + t \Delta \lambda_{nt}$
    
\textbf{until} Ax=b and $ \parallel r(x, \lambda) \parallel_2 \le \epsilon$
\end{algorithmic}
\end{algorithm}

\end{document}
